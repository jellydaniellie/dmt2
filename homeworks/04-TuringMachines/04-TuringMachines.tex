\documentclass[12pt]{article}
\usepackage[top=1in,bottom=1in,left=0.75in,right=0.75in,centering]{geometry}
\usepackage{fancyhdr}
\usepackage{epsfig}
\usepackage[pdfborder={0 0 0}]{hyperref}
\usepackage{palatino}
\usepackage{wrapfig}
\usepackage{lastpage}
\usepackage{color}
\usepackage{ifthen}
\usepackage[table]{xcolor}
\usepackage{graphicx,type1cm,eso-pic,color}
\usepackage{hyperref}
\usepackage{amsmath}
\usepackage{wasysym}


\hypersetup{
    colorlinks=true,
    linkcolor=blue,
    filecolor=magenta,      
    urlcolor=cyan,
    pdftitle={CS3120 - Homework 4},
    pdfpagemode=FullScreen,
    }

\urlstyle{same}

\def\course{CS 3120: Discrete Math and Theory II}
\def\homework{Turing Machines and Decidability}
\def\semester{Spring 2025}

\newboolean{solution}
\setboolean{solution}{false}

% add watermark if it's a solution exam
% see http://jeanmartina.blogspot.com/2008/07/latex-goodie-how-to-watermark-things-in.html
\makeatletter
\AddToShipoutPicture{%
\setlength{\@tempdimb}{.5\paperwidth}%
\setlength{\@tempdimc}{.5\paperheight}%
\setlength{\unitlength}{1pt}%
\put(\strip@pt\@tempdimb,\strip@pt\@tempdimc){%
\ifthenelse{\boolean{solution}}{
\makebox(0,0){\rotatebox{45}{\textcolor[gray]{0.95}%
{\fontsize{5cm}{3cm}\selectfont{\textsf{Solution}}}}}%
}{}
}}
\makeatother

\pagestyle{fancy}

\fancyhf{}
\lhead{\course}
\chead{Page \thepage\ of \pageref{LastPage}}
\rhead{\semester}
%\cfoot{\Large (the bubble footer is automatically inserted into this space)}

\setlength{\headheight}{14.5pt}

\newenvironment{itemlist}{
\begin{itemize}
\setlength{\itemsep}{0pt}
\setlength{\parskip}{0pt}}
{\end{itemize}}

\newenvironment{numlist}{
\begin{enumerate}
\setlength{\itemsep}{0pt}
\setlength{\parskip}{0pt}}
{\end{enumerate}}

\newcounter{pagenum}
\setcounter{pagenum}{1}
\newcommand{\pageheader}[1]{
\clearpage\vspace*{-0.4in}\noindent{\large\bf{Page \arabic{pagenum}: {#1}}}
\addtocounter{pagenum}{1}
\cfoot{}
}

\newcounter{quesnum}
\setcounter{quesnum}{1}
\newcommand{\question}[2][??]{
\begin{list}{\labelitemi}{\leftmargin=2em}
\item [\arabic{quesnum}.] {} {#2}
\end{list}
\addtocounter{quesnum}{1}
}


\definecolor{red}{rgb}{1.0,0.0,0.0}
\newcommand{\answer}[2][??]{
\ifthenelse{\boolean{solution}}{
\color{red} #2 \color{black}}
{\vspace*{#1}}
}

\definecolor{blue}{rgb}{0.0,0.0,1.0}

\begin{document}

\section*{\homework}


\question[3]{
Give implementation level descriptions for Turing Machines that decides the following two languages.
}

\begin{itemize}
	\item $\{w \ | \ w \ \text{contains an equal number of 0s and 1s} \}$
	\item $\{W = w\#w \ | \ w \in \{0,1\}^* \}$
\end{itemize}


\vspace{12pt}

\question[3]{
Let $M_{DFA} = \{A | \text{A is a DFA that does not accept any string with an odd number of 1's}\}$. Show that $M_{DFA}$ is decidable.
}

\vspace{12pt}

\question[3]{
Suppose we take the Turing Machine as defined in class, and we change the transition function to only allow the following: $\delta : Q \times \Gamma \rightarrow Q \times \Gamma \times \{S,R\}$. In other words, these machines can move the head right or stay in the same position but can never move left. Does this recognize the same set of languages as the normal \emph{Turing Machine}. If so, why? If not, what class of languages does it recognize?
}

\vspace{12pt}

\question[3]{
A \emph{Turing Machine with doubly-infinite tape} is an ordinary Turing Machine, except that the tape is infinitely indexed in both the left and right direction. Prove that this machine is equivalent in computational power to a traditional Turing Machine. 
}

\vspace{12pt}

\question[3]{
For this question, you will do five separate proofs. Prove that the class of \emph{Decidable Languages} is closed under \emph{union}, \emph{concatenation}, \emph{star}, \emph{complement}, and \emph{intersection}.
}

\vspace{12pt}

\question[3]{
Prove the following claim: Let $C$ be a language. Prove that $C$ is Turing-recognizable if and only if a decidable language $D$ exists such that $C=\{x | \exists y ((x,y) \in D)\}$
}

\vspace{12pt}

\question[3]{
Consider the following decision problem: "Given a Turing Machine \emph{M} and input \emph{w}, does \emph{M} ever move it's head to the left three times in a row?". Show, via reduction, that this problem is \emph{undecidable}.
}




\end{document}
