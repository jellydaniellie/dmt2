\documentclass[12pt]{article}
\usepackage[top=1in,bottom=1in,left=0.75in,right=0.75in,centering]{geometry}
\usepackage{fancyhdr}
\usepackage{epsfig}
\usepackage[pdfborder={0 0 0}]{hyperref}
\usepackage{palatino}
\usepackage{wrapfig}
\usepackage{lastpage}
\usepackage{color}
\usepackage{ifthen}
\usepackage[table]{xcolor}
\usepackage{graphicx,type1cm,eso-pic,color}
\usepackage{hyperref}
\usepackage{amsmath}
\usepackage{wasysym}
\usepackage{amsfonts}

\def\course{CS 3120: Discrete Math and Theory II}
\def\homework{Set Cardinality}
\def\semester{Fall 2023}

\newboolean{solution}
\setboolean{solution}{false}

% add watermark if it's a solution exam
% see http://jeanmartina.blogspot.com/2008/07/latex-goodie-how-to-watermark-things-in.html
\makeatletter
\AddToShipoutPicture{%
\setlength{\@tempdimb}{.5\paperwidth}%
\setlength{\@tempdimc}{.5\paperheight}%
\setlength{\unitlength}{1pt}%
\put(\strip@pt\@tempdimb,\strip@pt\@tempdimc){%
\ifthenelse{\boolean{solution}}{
\makebox(0,0){\rotatebox{45}{\textcolor[gray]{0.95}%
{\fontsize{5cm}{3cm}\selectfont{\textsf{Solution}}}}}%
}{}
}}
\makeatother

\pagestyle{fancy}

\fancyhf{}
\lhead{\course}
\chead{Page \thepage\ of \pageref{LastPage}}
\rhead{\semester}
%\cfoot{\Large (the bubble footer is automatically inserted into this space)}

\setlength{\headheight}{14.5pt}

\newenvironment{itemlist}{
\begin{itemize}
\setlength{\itemsep}{0pt}
\setlength{\parskip}{0pt}}
{\end{itemize}}

\newenvironment{numlist}{
\begin{enumerate}
\setlength{\itemsep}{0pt}
\setlength{\parskip}{0pt}}
{\end{enumerate}}

\newcounter{pagenum}
\setcounter{pagenum}{1}
\newcommand{\pageheader}[1]{
\clearpage\vspace*{-0.4in}\noindent{\large\bf{Page \arabic{pagenum}: {#1}}}
\addtocounter{pagenum}{1}
\cfoot{}
}

\newcounter{quesnum}
\setcounter{quesnum}{1}
\newcommand{\question}[2][??]{
\begin{list}{\labelitemi}{\leftmargin=2em}
\item [\arabic{quesnum}.] {} {#2}
\end{list}
\addtocounter{quesnum}{1}
}


\definecolor{red}{rgb}{1.0,0.0,0.0}
\newcommand{\answer}[2][??]{
\ifthenelse{\boolean{solution}}{
\color{red} #2 \color{black}}
{\vspace*{#1}}
}

\definecolor{blue}{rgb}{0.0,0.0,1.0}

\begin{document}

\section*{\homework}


\question[3]{
Prove that the set of rational numbers $\mathbb{Q}=\frac{a}{b}$ where $a \in \mathbb{Z}$ and $b \in \mathbb{Z}^+$ is countable. \emph{Hint: find a bijection between $\mathbb{Z} \times \mathbb{Z}^+$ and $\mathbb{N}$}
}

\vspace{12pt}

\question[3]{
Use a proof by diagonalization to show that the following set is uncountable:\\
\\
$F=\{ f:\mathbb{N} \rightarrow \mathbb{N} | (a>b) \rightarrow (f(a) > f(b)) \}$
\\
\\
In other words, this is the seet of all strictly increasing functions that map natural numbers to natural numbers. A function is strictly increasing if larger inputs are guaranteed to produce larger outputs.\\
\\
For example, $f(x)=x^2$ is strictly increasing since if $a>b$, then $a^2>b^2$. However, $f(x)=(x-5)^2$ is not strictly increasing since $1<2$ but $f(1) > f(2)$.
}

\vspace{12pt}

\question[3]{
Prove that every subset of the natural numbers $\mathbb{N}$ is countable.
}

\vspace{12pt}

\question[3]{
Prove the following claim: If $A$ is a countably infinite set (i.e., $|A|=|\mathbb{N}|$) and $B$ is a finite set (i.e., $|B|=n | n \in \mathbb{N}$), then $A \cup B$ is also countable.
}

\vspace{12pt}

\question[3]{
Prove the following claim: If $A$ is a countably infinite set (i.e., $|A|=|\mathbb{N}|$) and $B$ is a also a countably infinite set (i.e., $|B| = |\mathbb{N}|$), then $A \cup B$ is also countable.
}

\vspace{12pt}

\end{document}
