\documentclass[12pt]{article}
\usepackage[top=1in,bottom=1in,left=0.75in,right=0.75in,centering]{geometry}
\usepackage{fancyhdr}
\usepackage{epsfig}
\usepackage[pdfborder={0 0 0}]{hyperref}
\usepackage{palatino}
\usepackage{wrapfig}
\usepackage{lastpage}
\usepackage{color}
\usepackage{ifthen}
\usepackage[table]{xcolor}
\usepackage{graphicx,type1cm,eso-pic,color}
\usepackage{hyperref}
\usepackage{amsmath}
\usepackage{amsfonts}
\usepackage{wasysym}

\def\course{CS 3120: Discrete Math and Theory II}
\def\homework{What is a computer? Proof Techniques}
\def\semester{Fall 2024}

\newboolean{solution}
\setboolean{solution}{false}

% add watermark if it's a solution exam
% see http://jeanmartina.blogspot.com/2008/07/latex-goodie-how-to-watermark-things-in.html
\makeatletter
\AddToShipoutPicture{%
\setlength{\@tempdimb}{.5\paperwidth}%
\setlength{\@tempdimc}{.5\paperheight}%
\setlength{\unitlength}{1pt}%
\put(\strip@pt\@tempdimb,\strip@pt\@tempdimc){%
\ifthenelse{\boolean{solution}}{
\makebox(0,0){\rotatebox{45}{\textcolor[gray]{0.95}%
{\fontsize{5cm}{3cm}\selectfont{\textsf{Solution}}}}}%
}{}
}}
\makeatother

\pagestyle{fancy}

\fancyhf{}
\lhead{\course}
\chead{Page \thepage\ of \pageref{LastPage}}
\rhead{\semester}
%\cfoot{\Large (the bubble footer is automatically inserted into this space)}

\setlength{\headheight}{14.5pt}

\newenvironment{itemlist}{
\begin{itemize}
\setlength{\itemsep}{0pt}
\setlength{\parskip}{0pt}}
{\end{itemize}}

\newenvironment{numlist}{
\begin{enumerate}
\setlength{\itemsep}{0pt}
\setlength{\parskip}{0pt}}
{\end{enumerate}}

\newcounter{pagenum}
\setcounter{pagenum}{1}
\newcommand{\pageheader}[1]{
\clearpage\vspace*{-0.4in}\noindent{\large\bf{Page \arabic{pagenum}: {#1}}}
\addtocounter{pagenum}{1}
\cfoot{}
}

\newcounter{quesnum}
\setcounter{quesnum}{1}
\newcommand{\question}[2][??]{
\begin{list}{\labelitemi}{\leftmargin=2em}
\item [\arabic{quesnum}.] {} {#2}
\end{list}
\addtocounter{quesnum}{1}
}


\definecolor{red}{rgb}{1.0,0.0,0.0}
\newcommand{\answer}[2][??]{
\ifthenelse{\boolean{solution}}{
\color{red} #2 \color{black}}
{\vspace*{#1}}
}

\definecolor{blue}{rgb}{0.0,0.0,1.0}

\begin{document}

\section*{\homework}

\question[3]{
Consider the formal descriptions of each set below. For each, write a short informal English description of each set.
}

\begin{itemize}
	\item $\{n | \exists_{m \in \mathbb{N}} : n=2m\}$ 
	\item $\{n | \exists_{m \in \mathbb{N}} : n=2m \wedge \exists_{k \in \mathbb{N}} : n=3k \}$ 
	\item $\{ w \ | \ w \in \{0,1\}^* \wedge w=w^R \}$ \emph{**Note that $w^R$ is the reverse string of $w$ (e.g., 001 becomes 100)}
	\item $\{ n | n \in \mathbb{Z} \wedge n=n+1 \}$
\end{itemize}

\vspace{12pt}

\question[3]{
Similarly, for each informal description of the following languages, write out a formal version of the same set (in similar detail to what you see in the question above).
}

\begin{itemize}
	\item The set containing all integers greater than 5
	\item The set containing all bitstrings that contain 010 somewhere within them
	\item The set containing all bitstrings that are odd
\end{itemize}

\vspace{12pt}

\question[3]{
\textbf{Find and describe the error in the following direct proof that 2=1:} Consider the equation $a=b$. Multiply both sides by $a$ to obtain $a^2=ab$. Subtract $b^2$ from both sides to get $a^2-b^2=ab-b^2$. Now factor each side, $(a+b)(a-b)=b(a-b)$ and divide each side by $(a-b)$ to get $a+b=b$. Finally, let $a=b=1$ and substitute to get $2=1$.
}

\vspace{12pt}

\question[3]{
\textbf{Find and describe the error in the following inductive proof that all pairs of UVa students have worked together on at least one project:} First, order the students alphabetically by computing id.\\
\\
\textbf{Base case}: consider $n=1$. With one student, the theorem is trivially true. There is only one student and thus they have worked with every other student on at least one project.\\
\\
\textbf{Inductive hypothesis}: Now assume that the claim is true for any arbitrary number of students $n \leq k$.\\
\\
\textbf{Inductive step}: Now try to prove it still holds for $n=k+1$ students. Take the $k+1$ students and arbitrarily remove one. By the inductive hypothesis, the remaining $k$ students have all worked on at least one project together. Now do the same but remove a different single student. Again, by the inductive hypothesis the remaining $k$ students have all worked on at least one project together. Do this one last time with a third unique student being remove. Because we removed a different single student each of the three times, all of the $k+1$ students must have worked together on at least one project because every pair of students was in at least one of the subsets of size $k$.\\
\\
In your description, make sure to emphasize not just the error in logic, but the properties of inductive proofs that must be carefully followed in order for the proof to be valid. In other words, we are NOT looking for a purely intuitive answer.}

\vspace{12pt}


\question[3]{
Consider a graph $G=(V,E)$ where $|V| \% 2 = 0 \wedge |V| \geq 2$. Prove that if the degree of each node is at least $\frac{|V|}{2}$, then $G$ must be connected. \emph{$G$ does not contain any self-directed edges and $G$ is undirected.}
}

\vspace{12pt}


\question[3]{
Suppose I build a new computing machine that can be programmed to recognize a lot of different functions! It is called the \emph{Flogrammable Device}. In order to program this machine, you can type out your program on a \emph{tape}, but this tape can only hold ten characters. Each character is from the alphabet $\Sigma=\{a,b,c,d,0,1\}$ and any combination of these 10 characters is a valid program. More precisely, a program is a String $P = p_1p_2...p_{10} \ | \ \forall_i \ p_i \in \Sigma$. You CANNOT have fewer than 10 characters or the code will not compile.\\
\\
Now suppose that you read online that somehow, there are 100,000,000 important functions that need to be computed by the \emph{Flogrammable Device} for it to cover all important functionality. Can we program each of the 100,000,000 functions on this machine? How do you know or not?
}


\end{document}