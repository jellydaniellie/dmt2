\documentclass[12pt]{article}
\usepackage[top=1in,bottom=1in,left=0.75in,right=0.75in,centering]{geometry}
\usepackage{fancyhdr}
\usepackage{epsfig}
\usepackage[pdfborder={0 0 0}]{hyperref}
\usepackage{palatino}
\usepackage{wrapfig}
\usepackage{lastpage}
\usepackage{color}
\usepackage{ifthen}
\usepackage[table]{xcolor}
\usepackage{graphicx,type1cm,eso-pic,color}
\usepackage{hyperref}
\usepackage{amsmath}
\usepackage{wasysym}
\usepackage{amsfonts}

\def\course{CS 3120: Discrete Math and Theory II}
\def\homework{Regular Languages}
\def\semester{Fall 2024}

\newboolean{solution}
\setboolean{solution}{false}

% add watermark if it's a solution exam
% see http://jeanmartina.blogspot.com/2008/07/latex-goodie-how-to-watermark-things-in.html
\makeatletter
\AddToShipoutPicture{%
\setlength{\@tempdimb}{.5\paperwidth}%
\setlength{\@tempdimc}{.5\paperheight}%
\setlength{\unitlength}{1pt}%
\put(\strip@pt\@tempdimb,\strip@pt\@tempdimc){%
\ifthenelse{\boolean{solution}}{
\makebox(0,0){\rotatebox{45}{\textcolor[gray]{0.95}%
{\fontsize{5cm}{3cm}\selectfont{\textsf{Solution}}}}}%
}{}
}}
\makeatother

\pagestyle{fancy}

\fancyhf{}
\lhead{\course}
\chead{Page \thepage\ of \pageref{LastPage}}
\rhead{\semester}
%\cfoot{\Large (the bubble footer is automatically inserted into this space)}

\setlength{\headheight}{14.5pt}

\newenvironment{itemlist}{
\begin{itemize}
\setlength{\itemsep}{0pt}
\setlength{\parskip}{0pt}}
{\end{itemize}}

\newenvironment{numlist}{
\begin{enumerate}
\setlength{\itemsep}{0pt}
\setlength{\parskip}{0pt}}
{\end{enumerate}}

\newcounter{pagenum}
\setcounter{pagenum}{1}
\newcommand{\pageheader}[1]{
\clearpage\vspace*{-0.4in}\noindent{\large\bf{Page \arabic{pagenum}: {#1}}}
\addtocounter{pagenum}{1}
\cfoot{}
}

\newcounter{quesnum}
\setcounter{quesnum}{1}
\newcommand{\question}[2][??]{
\begin{list}{\labelitemi}{\leftmargin=2em}
\item [\arabic{quesnum}.] {} {#2}
\end{list}
\addtocounter{quesnum}{1}
}


\definecolor{red}{rgb}{1.0,0.0,0.0}
\newcommand{\answer}[2][??]{
\ifthenelse{\boolean{solution}}{
\color{red} #2 \color{black}}
{\vspace*{#1}}
}

\definecolor{blue}{rgb}{0.0,0.0,1.0}

\begin{document}

\section*{\homework}

\question[3]{
Draw out \emph{DFA}s (not \emph{NFA}s) for each of the following languages. For some of these, a small hint is provided. Your goal is to construct a \emph{DFA} with as few states as possible (just like how we prefer to write succinct code when possible). For all of these, let $\Sigma = \{a,b\}$
}

\begin{itemize}
	\item $\{w \ | \ w \text{ does not contain the substring } abba \}$ (\emph{*Hint: Draw out the DFA for a simpler language that DOES contain ab and then try to change that machine slightly.})
	\item $\{w \ | \ w \text{ contains BOTH the substrings ab and ba}\}$
	\item $\{w \ | \ w \in a^*b^*a^* \}$
	\item $\{w \ | \ w \neq ab \wedge w \neq bb \}$
	\item $\{w \ | \ w \in a^iw \ | \ i \in \mathbb{N}, w \in \{a,b\}^*, \text{w contains at least i a's} $ (\emph{*Hint: This one LOOKS not regular but it actually is. Can you figure out why?})
\end{itemize}

\vspace{12pt}

\question[3]{
Prove that regular languages are closed under \emph{intersection}. Do this by starting with \emph{DFA}s for two regular languages $A$ and $B$, and describe how to construct a new \emph{DFA} for $A \cap B$
}

\vspace{12pt}

\question[3]{
Prove that regular languages are closed under \emph{complement}. Do this by starting with a \emph{DFA} for a regular language $A$, and describe how to construct a new \emph{DFA} for $\bar{A}$.
}

\vspace{12pt}

\question[3]{
For any string $w = w_1w_2,...,w_n$, let $w^R$ be the reverse of string $w$ (i.e., $w^R=w_n,...,w_2,w_1$). Prove that if a language $A$ is regular, then the language $A^R = \{w^R \ | \ w \in A\}$ is also regular.
}


\vspace{12pt}

\question[3]{
Use the pumping lemma to show that the following languages are not regular. 
}

\begin{itemize}
	\item $A= \{ 0^*0^n1^n1^* \ | \ n \geq 0\}$
	\item $B= \{ w0^*011^*w \ | \ w \in \{0,1\}^* \}$
\end{itemize}

\vspace{12pt}

\question[3]{
Find and describe the error that exists in the following proof. The proof attempts to show that $0^*1^*$ is not regular, when in fact it is:
\\
\\
\emph{
Assume, for sake of contradiction, that $0^*1^*$ is regular. We select an element from this language that is greater than the pumping length $p$. We select $0^p1^p$. In class, when proving that $0^n1^n$ was not regular, we showed that $0^p1^p$ cannot be pumped. Therefore, $0^*1^*$ is not regular.
}
}

\vspace{12pt}


\end{document}
